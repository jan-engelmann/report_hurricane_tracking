% Table of contents

 \setcounter{tocdepth}{2}
 \tableofcontents

%---------------------------------------------------------------------------
% Abstract


\chapter*{Abstract}
 \addcontentsline{toc}{chapter}{Abstract}
Tropical Cyclone (TC) tracking in simulation data requires parameter thresholds that specify the expected intensity of these characterising variables. This results in assumptions for a specific climate model that may or may not lead to a successful tracking of TCs. With the purpose of being able to run the algorithm with different parameter assumptions, an existing algorithm was optimised and a 60-fold speedup reached. This enabled experimentation with numerous different threshold combinations. It was found that correctly adjusting the warm core criterion is of central importance since it balances the unwanted tracking of noise with the desired early discovery of tropical cyclones. Furthermore, including a requirement in regards to the minimum sea surface temperature during TC genesis made the algorithm more robust. The analysis of the vorticity criterion suggested further investigation of its effectiveness. Finally, matching tracks across threshold combinations significantly improved the understanding of parameter interplay and the tracking of TCs across their lifetime.