\chapter{Introduction}\label{sec:introduction}
Tropical cyclones also known as Hurricanes or Typhoons are storms of extreme nature in many regards. Not only are they the most deadly and expensive natural catastrophes in the United States but also their physics is quite challenging with many open questions remaining.\cite{emanuel-summ}
Finally they are very likely to play a crucial role in the global heat balance and moisture circulation.\cite{moisture-transport}\cite{global-heat}

\section{Impact on society}\label{sec:society}
While tropical cyclones form and intensify above the ocean, they have the largest impact on society during landfall. The damage happens due to a combination of strong winds and catastrophic storm surges. On average hurricanes inflict normalized damages of about \$10-billion/year in the United States.\cite{damage-norm} A single strong storm can cause thousands of deaths. Hurricane Katrina e.g. in 2005 took the lives of over 1200 people.\cite{hurr-2005}
Due to these enormous implications for society and opportunities to save lives and money, active research is happening on tropical cyclone impact reduction. With an unsure impact of climate change on tropical cyclone frequency and intensity and a trend of urbanization on the American east-coast, improving the understanding of tropical cyclones is of great importance.

\section{Underlying Physics}\label{sec:physics}
Tropical Cyclones can be classified using the Saffir-Simpson wind scale. It is defined by the maximum wind observed in the cyclone. Due to their characteristic structure this wind usually occurs at the eyewall. This choice is motivated by the strong correlation between wind speeds and the inflicted damages.\cite{simpson} The exact categorization can be seen in table~\ref{tab:simpson-scale}.

\begingroup
\setlength{\tabcolsep}{10pt} % Default value: 6pt
\renewcommand{\arraystretch}{1.5} % Default value: 1
\begin{table}[hbt!]
	\centering
	\begin{tabular}{|c|c|c|c|c|}
		\cline{1-2} \cline{4-5}
		\multicolumn{2}{|c|}{\textbf{Tropical Cyclones}} &                      & \multicolumn{2}{c|}{\textbf{Other tropical low pressure systems}}                                                  \\ \cline{1-2} \cline{4-5}
		category                                         & wind speed {[}m/s{]} &                                                                   & name, category          & wind speed {[}m/s{]} \\ \cline{1-2} \cline{4-5}
		1                                                & 33--42               &                                                                   & tropical depression, -1 & $\leq$ 17            \\
		2                                                & 43--49               &                                                                   & tropical storm, 0       & 18--32               \\
		3                                                & 50--58               &                                                                   &                         &                      \\
		4                                                & 58--70               &                                                                   &                         &                      \\
		5                                                & $\geq$ 70            &                                                                   &                         &                      \\ \cline{1-2} \cline{4-5}
	\end{tabular}
	\caption{Simpson scale defined by 1-minute maximum sustained winds \cite{simpson}. The category number of the other tropical low pressure systems was assigned by the author and will be used in the results section.}
	\label{tab:simpson-scale}
\end{table}
\endgroup

Their occurrence frequency is displayed in Fig.~\ref{fig:cat-climatology}
\begin{figure}[ht]
	\centering
	\includegraphics[width=0.7\textwidth]{img/cum-average-cat.png}
	\caption{Average cumulative number of storms in the Atlantic. Named systems are mostly tropical storms but can also be tropical depressions.\cite{climatology}}
	\label{fig:cat-climatology}
\end{figure}

Tropical Cyclones are rotating low pressure systems that exhibit a warm core. They rotate around a pressure minimum with low wind speeds which is enclosed by the eyewall which in turn is the radius of maximum wind. The rotation around the center happens in different bands of updraft that are alternated with rain bands. Finally the the warm updraft in the center cools down on its way to the top. Resulting in a cold outflow at the top of the storm. This structure is depicted in Fig.~\ref{fig:tc-structure}.
\begin{figure}[ht]
	\centering
	\includegraphics[width=0.8\textwidth]{img/hurricane-structure.png}
	\caption{Structure of a Tropical Cyclone in the Northern hemisphere~\cite{hurricane-structure}}
	\label{fig:tc-structure}
\end{figure}
In order for a storm to develop a number of TC genesis criteria have to be met. Not all of them have to be satisfied but they do offer a good indicator for the probability of storm formation.
The criteria are as follows:
\begin{itemize}
	\item sea surface temperature above 26.5\degree C to at least a depth of 50m
	\item sufficiently moist mid-troposphere for deep convection
	\item Appreciable moisture flux at the ocean-air interface to sustain a conditionally unstable thermodynamic environment (Montgomery \& Smith, 2017)
	\item A distance of at least 5 from the equator, so that the Coriolis effect is strong enough to initiate the cyclone's rotation (Palm\'{e}n, 1948).
	\item A pre-existing weather disturbance with sufficient vorticity and convergence, e.g. a tropical easterly wave.
	\item Low vertical wind shear between the surface and the upper troposphere
\end{itemize}
\cite{lohmann-storms}

\section{Previous work on TC Tracking}\label{sec:tracking}
A large body of work exists on tracking metrological phenomena. In this context tracking can be understood as following the same physical object across times.\newline
For instance an effort was conducted to compare different extratropical cyclone tracking algorithms in ~\cite{extratropical}. While extratropical cyclones differ significantly from tropical cyclones for example in regards to their occurrence frequency and physical structure, there are some common themes that apply to both tracking endeavours. The study found that the 15 compared algorithms disagree strongly on the total number of cyclones and the detection of weak cyclones. Furthermore, the tracking methods agree best for stronger cyclones. It will be shown in Sec.~\ref{sec:results} of this report that all of these facts were also found to apply to the tracking of TCs.\newline
Another work compared the predictions of two TC tracking schemes for several idealised climate simulations \cite{comp-climate-schemes}. The authors find a strong dependence of the results on the used parameter thresholds. They furthermore distinguish between algorithms using experimentally motivated parameter thresholds and those using deviations from the surrounding mean.
