\chapter{Conclusion}\label{sec:conclusion}
This project significantly improved the speed and efficiency of a TC tracking algorithm. For a single ensemble member and parameter combination the speedup is sixtyfold. However, because the new algorithm is I/O bound, it really wins over the previous one when many parameter combinations are tested on the same simulation run. A tracking run including 600 parameter combinations and 20 members is 30000-times faster than running the previous algorithm for each parameter combination and member separately. \newline
An exploratory data analysis revealed interesting aspects of the different tracking criteria as outlined before. The capability to quickly try different parameter combinations and the already produced tracking data empowers a wide range of further analysis steps. One idea would be to define a metric that measures the quality of the produced tracking data. This metric could be based on statistics from reanalysis data.\newline
A very illuminating endeavour was the matching of tracks across parameter combinations. This lead to an improved understanding of the influence of each parameter and their interplay on the result. Furthermore, it showed the way to robust tracking across TC development phases.\newline
On the technical side, the use of a high level library made for parallel computation could lead to an even better use of the computing resources. From initial screenings, the Dask library \cite{dask} seems promising. Especially the reading of data could be improved, since now the 20 different processes that track TCs in the different simulation runs access different files on the same disk. Using one process that reads data and feeds it to other process that analyse it would reduce the disk handover time between processes.\newline
Finally, combining the ideas presented in this conclusion, would lead to a procedure that can be robustly applied to different simulation and reanalysis data, requiring only little calibration.
