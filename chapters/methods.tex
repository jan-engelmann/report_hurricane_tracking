\chapter{Data and Methods}\label{sec:methods}
The purpose of this project was to evaluate the parameter thresholds for tracking TCs in ICON simulation data. In the following section the settings used for generating the simulation data and the algorithm itself will be explained.
\section{Simulation Data}\label{sec:data}
\subsection*{ICON Model}
The Icosahedral Nonhydrostatic Model (ICON) is developed by the German Weather Service (DWD), the Max Planck Institute for Meteorology and several partner institutes. As the name implies the grid is in a first step generated by mapping the earths surface to an Icosahedron (platonic solid with 20 equilateral triangles as faces). The faces are then split up into smaller triangles in order to achieve the desired resolution. The model then delivers all typical meteorological quantities on this grid.
The model uses the fully compressible and non-hydrostatic version of the Euler equations for the fundamental transport processes. Physical processes which cannot be resolved with the grid size are then parametrised by using complex functions that take into account the grid box mean values of the model variables. Physical processes which need to be treated in this way include solar and thermal radiation, cloud microphysics and turbulent transfer above the earth's surface.\cite{dwd-icon}

\subsection*{Analysed simulation output}
The North Atlantic basin with its surrounding area ( TODO xx1 - xx2 lon and yy1 - yy2 lat) % TODO find out exact domain 
was simulated from the beginning of May until the beginning of December. The output data is saved for every six hours.
Two different kinds of time lagged ensembles were used. Both used ERA5 reanalysis data for the initial weather state and the monthly boundary conditions. The first however, labeled "ref", used the first day of each month at 6:00 a.m. (TODO !confirm with Bernhard ) % TODO confirm time lagged ensemble with bernhard
as the monthly boundary conditions. The second ensemble, labeld "rm", used monthly rolling mean boundary conditions based on the same reanalysis data.
The members within each of these two ensembles were created by varying the initial weather state of the simulation. Namely each of the first 10 days of May were used as initial conditions for a separate simulation run. 
The output data from each run is mapped from the icosahedral to a Latitude-Longitude grid in order to facilitate distance calculations.
Finally this results in 20 separate simulation runs that can be searched for TCs.
\section{Algorithm}
An existing algorithm developed by Bernhard Enz and inspired by \cite{orig-tracking} was improved in regards to runtime, robustness and readability. The gained speed was used in order to run the algorithm on the same simulation data but with different threshold values that decide wether a TC was found or not. By comparing the different results reasonable thresholds and the importance of the different criteria were determined.
\subsection{TC candidate search}
The algorithm consists of two steps. In the first step all time-steps are scanned separately for TC candidates. The whole domain is searched for TCs by applying the following criteria:
\subsubsection*{Sea Level Pressure Minimum}
As outlined in Sec.~\ref{sec:physics} TCs are low pressure systems. In fact some of the lowest pressures on earth were measured inside the eyes of TCs. Therefore the first step to finding TC candidates for a specific time-step is locating the sea level pressure minima. Here enormous speed up was achieved by replacing the previous manual minimum finding algorithm with a vectorised version from the image processing library scikit-image~\cite{scikit-image}. This function finds the local minima and requires them to be a certain distance apart. The stronger minimum is kept if two candidates are within the specified distance \textbf{slpdis}. The resulting minima are then further analysed.
\subsubsection*{Minimal Vorticity Threshold}
If the vorticity at a pressure minimum is below the minimum threshold \textbf{vormin} it is discarded as a potential TC candidate. 
\subsubsection*{Warm Core Criterion}
O
 the parameters \textbf{temdif} and \textbf{temdis}. They correspond to the required temperature difference between the 

\begin{enumerate}
    \item Explain Criteria
    \item mention long running time
\end{enumerate}
\begin{enumerate}
    \item Vectorization of Operations especially slp minimum finding
    \item Parallelization of Experiments and Members
    \item variation of parameters
    \item ? reordering of criteria (hope there will be time)
\end{enumerate}
\subsection{Creating TC tracks from previously found TC candidates}

\section{Tracking Data Analysis}
\subsection{Algorithm output data format}
\begin{enumerate}
    \item Quickly explaining what the algorithm outputs
\end{enumerate}
\subsection{Analysis of variation of parameters}
Here I will discuss the analyses performed for different parameter combinations.
would be grateful for some advice on how to evaluate validity of a parameter value and what analyses are interesting.
\subsubsection{slpdis}
\subsubsection{vormin}
\subsubsection{temdif}
\subsubsection{temdis}
\subsubsection{winddis}
\subsubsection{maxhgt}




